\documentclass[a4paper]{article}

%%  [2015/03/18 v1.1]
%%  Template for general use in relation with STNB.
%% For the latest version of this STNB proceedings class, see:
%%         https://github.com/hds/stnb-proceedings
%%
%%  It can be used for the announcements of any activity related with the Seminari de Teoria de Nombres de Barcelona all year round.
%%
%%  The first part of the document can be updated in order to fix the year, the edition.
%%

\usepackage[utf8]{inputenc}
\usepackage[T1]{fontenc}
\usepackage{lmodern}

\usepackage{amsmath}
\usepackage{amssymb}

\usepackage{fullpage}
\usepackage{graphicx}
\usepackage[pdftex,colorlinks,linkcolor=blue,citecolor=blue,urlcolor=blue]{hyperref}


\setlength{\parindent}{0pt}
\setlength{\parskip}{1.5ex}




\begin{document}
\pagestyle{empty} 


\noindent
\begin{minipage}{0.75\textwidth}
    \fontsize{12}{10} \selectfont
	\sffamily 
\textbf{SEMINARI DE TEORIA DE NOMBRES DE BARCELONA
}
   \fontsize{11.5}{14} \selectfont
    \sffamily 
    \textbf{BARCELONA NUMBER THEORY SEMINAR (UB-UAB-UPC)}
	
	
	
\end{minipage}
%
\begin{minipage}{0.25\textwidth}
\hfill\includegraphics[width=30mm]{stnb-color}
\end{minipage}



\vspace{2truecm}




\fontsize{14}{10} \selectfont
    \sffamily 

\textbf{Divendres 20 de març, 2015,  a les 12:15h}
\\ 
Facultat de Matemàtiques i Estadística (UPC) 
\\
C/ Pau Gargallo 5, Barcelona

\vspace{2truecm}



\fontsize{16}{16} \selectfont


\begin{center}
\textbf{Cotas uniformes de torsión de curvas elípticas \\
 en términos de ramificación}

\fontsize{14}{18} \selectfont
\textbf{A. Lozano (Univ. Connecticut)}
\end{center}

\vspace{1truecm}

\fontsize{12}{10} \selectfont

{\large \textbf{Abstract:}

Sea $d\geq 1$ un entero. Sea $F$ un cuerpo de números de grado
$d$, sea $E/F$ una curva elíptica y sea $E(F)_\text{tors}$ el subgrupo de
torsión de $E(F)$. En 1996, Merel demostró la "conjetura de la cota
uniforme", i.e., existe una constante $B(d)$, que sólo depende de $d$ pero
no del cuerpo $F$ elegido, ni de la curva $E/F$, tal que el orden de
$E(F)_\text{tors}$ está acotado por $B(d)$. Además, Merel dio una cota
(exponencial en $d$) para el mayor primo que puede aparecer como un
divisor del orden de $E(F)_\text{tors}$. En 1996, Parent demostró otra
cota (también exponencial en $d$) para la mayor potencia de $p$ que puede
aparecer como orden de un punto de torsión en $E(F)_\text{tors}$, aunque
se conjetura que existe una cota para el orden de $E(F)_\text{tors}$ que
es polinómica en $d$. En esta charla demostraremos que bajo ciertas
hipótesis hay una cota lineal para la mayor potencia de $p$ que puede
aparecer como orden de un punto de torsión sobre $F$, la cual, de hecho,
es lineal en el máximo índice de ramificación de un ideal primo del anillo
de enteros de $F$ sobre $(p)$.


\vspace{2truecm}

More information is available at \href{http://stnb.cat}{http://stnb.cat}.

\end{document}

